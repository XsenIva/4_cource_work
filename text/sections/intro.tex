В последнее время компании все чаще предпочитают легкие, быстырые и универсальные 
web-приложения для своих услуг, взмен тяжеловесным дестктопным. А из этого 
вытекает появление большого количества фремворков и библиотек для разработки и 
поддержки web-приложений, которые позволяют реализовать все, что необходимо для 
функционирования современного web-приложения.
Поэтому одной из сложностей является выбор подходящего для целей приложения фреймворка, 
а для правильного выбора нужно понимать достоинства и недостатки каждого. В этой работе я 
рассмотрю популярные среди разработчиков фреймворки с точки зрения безопасности, 
расстмотрю их функционал и оценю его работу.
