Фреймворк -- это динамически пополняемая библиотека языка программирования, в 
которой собраны его базовые модули. Фреймворки создаются для упрощения процессов разработки приложений, сайтов, сервисов.
Чтобы не писать модуль в приложении с нуля, гораздо проще обратиться к готовым шаблонам фреймворков, которые и формируют 
рабочую среду разработчика.

Архитектура почти всех фреймворков основана на декомпозиции нескольких отдельных слоев (приложения, модули и т. д.) проекта.
Это означает, что можно расширять функциональность приложения исходя из потребностей и использовать измененную версию вместе 
с кодом фреймворка или задействовать сторонние приложения. Такая гибкость является одним из одним из ключевых преимуществ 
использования фреймворков.

Рассмотрим в общем несколько популярных фреймворков и их особенности.

\subsection{ASP.NET (.NET)}
 
Учитывая популярность в свое время технологий .NET, стоит поговорить про такой фреймворк, как ASP.NET -- это набор технологий 
в составе .NET Framework, которые позволяют создавать
web-приложения и сервисы на основе Microsoft.NET с использованием любых поддерживаемых ей языков. В отличие от 
web-страниц, которые представляют собой сочетание статического HTML
и сценариев, платформа использует скомпилированные страницы, которые управляются событиями. Но в отличие от десктопных приложений, эти 
скомпилированные страницы создают информацию, отправляемую клиентам с использованием языков
разметки наподобие HTML и XML. Это позволяет разработчикам создавать приложения, защищая
при этом интерфейс пользователя под управлением разных операционных систем.

Основная концепция безопасности, которая есть в .NET -- прежде всего это ролевая модель безопасности. Она подразумевает 
два основных режима работы. Первый это создание пользователей и ролей, которые не 
зависят от ролей ОС Windows. Такая модель удобна, когда все разграничение прав внутри приложения ведется именно с помощью
ролей. Все это никоим образом не связано с учетными записями в ОС. Например, это доступность
каких-либо компонентов веб приложения в зависимости от заданной роли пользователя.

Второй режим -- жесткая привязка ролей в приложении к учетным записям в Windows. Обычно подобную модель безопасности 
можно встретить в веб-приложениях, работающих во внутренней ё 2среде и тесно связанных с инфраструктурой Active Directory.

Аутентификация в ASP.NET приложениях обычно реализуется или с помощью аутентификации Windows или с помощью форм. Первый
вариант построен на использовании штатных средств операционной системы . В каждом случае пользователь предъявляет некий
аналог “удостоверения” – в первом случае это SID (Security Identifier), а во втором случае формируется так называемый билет,
который затем сохраняется в cookie. 

Для работы с базой данных используется объектно-ориентированная технология доступа к данным -- Entity Framework (EF), которая 
является object-relational mapping (ORM) решением для .NET Framework от Microsoft. Изначально с самой первой версии Entity Framework поддерживал подход 
Database First, который позволял по готовой базе данных сгенерировать edmx модель данных файла. Затем эта модель использовалась для 
подключения к базе данных. Позже был добавлен подход Model First. Он позволял создать вручную с помощью визуального 
редактора модель, и по ней создать базу данных. Предпочтительным подходом стал Code First, в котором сначала пишется код 
модели, а затем по нему генерируется база данных.

\textbf{Положительные стороны:}
\begin{itemize}
  \item[-] Возможность разработки крупных проектов со сложной архитектурой;
  \item[-] Совместимость с несколькими операционными системами; 
  \item[-] Разработка на разных языках;
\end{itemize}


\textbf{Отрицательные стороны:}
\begin{itemize}
  \item[-] Проблемы с одновременным доступом большого числа пользователей;
  \item[-] Используется компиляция, при незначительных нагрузках сервисы работают медленнее;  
  \item[-] При разработке приложений используются лицензированные инструменты, что приводит к удорожанию продукта.
\end{itemize}


\subsection{Laravel (Php)}
На конец 2023 года, по статистике GitHub, Php занимает 7 место по популярности среди всех существующих языков разработки[6].
Так или иначе этот язык присутствует в 79,2 \% от общего числа всех вебсайтов в интернете, не удивительно, ведь он стоял у истоков интернета.
Один из популярнейших фреймворков Php -- Laravel считается довольно простым для входа среди всех Php-фреймворков, т.к делает процесс разработки веб-сайтов проще и быстрее из-за
простой обработки кода, и множества встроенных модулей. Что важно фреймворк содержит пакет безопасности, который позволяет повысить защищенность приложения.

Внутри фреймворка Laravel есть свой ORM -- Eloquent. В библиотеке ORM помимо стандартных CRUD-операций можно выделить наличие методов доступа,
мутаторов, безопасное удаление, направления областей запросов, построения отношений, построение взаимодействия на основе событий. 
Eloquent организует работу с базой через представление таблиц в виде <<моделей>>, через которы и осуществляется доступ и манипуляции с данныими.

\textbf{Положительные стороны:}
\begin{itemize}
  \item[-] простая интеграция платежных шлюзов;
  \item[-] встроенные пакеты шифрования, основанные на OpenSSL в системе алгоритмов AES-256 и AES-128; 
  \item[-] наличие встроенных шаблонизаторов Blad;
  \item[-] частая обновляемость и поддержка.
\end{itemize}

\textbf{Отрицательные стороны:}
\begin{itemize}
  \item[-] большой объем файлов и зависимостей, что может навредить производительности;
  \item[-] несовместимость обновлений фреймворка, что может привести к конфликтам внутри проекта.
\end{itemize}
   
    
Некоторые пункты, которые реализует фреймворк в плане безопасности.

\textbf{Токены CSRF}

Laravel автоматически генерирует «токен» CSRF для каждой активной пользовательской сессии, управляемой приложением.
Он используется для проверки, что аутентифицированный пользователь не является злоумышленником.

\textbf{Защита от XSS}

Laravel автоматически экранирует выводимые данные, чтобы предотвратить XSS-атаки. Чтобы безопасно выводить данные, используется Blade-синтаксис (шаблонизатор).

\textbf{Защита от SQL}

Существует несколько различных способов защиты от SQL-инъекций.


\subsection{Django (Python)}
Высокоуровневый фреймворк, который является не только быстрым решением в веб-разработке, включающим все необходимое для 
качественного кода и прозрачного написания, но и также удобным для разработчиков.
В Django реализован принцип DRY — Don't Repeat Yourself (не повторяйся). То есть при использовании Django не нужно 
несколько раз переписывать один и тот же код. Фреймворк позволяет создавать сайт из компонентов. Благодаря этому сокращается 
время создания сайтов.

Фреймворк справляется с большим количеством задач и повышенными нагрузками, также подходит для создания алгоритмических 
генераторов, платформ для электронных рассылок, систем верификации, платформ для анализа данных и сложных вычислений, машинного обучения.
У фреймворка есть своя ORM, которая автоматически передает данные из БД в объекты,
которые используются в коде приложения, включает механизмы предотвращения распространенных атак вроде SQL-инъекций 
и подделки межсайтовых запросов.

\textbf{Положительные стороны:}
\begin{itemize}
 \item[-] масса различных библиотек;
 \item[-] подробная документация и очень развитое сообщество;
 \item[-] возможность масштабирования по мере необходимости.
\end{itemize}

\textbf{Отрицательные стороны:}
\begin{itemize}
 \item[-] нет поддержки WebSockets;
 \item[-] Django ORM сегодня значительно уступает последней SQLAlchemy.
\end{itemize}

Некоторые пункты, которые реализует фреймворк в плане безопасности.

\textbf{Внедрение SQL}

Это предотвращается благодаря API QuerySet, который не делает никаких действий с базой,
пока набор запросов к ней не будет обработан. 

\textbf{Параметризация}

Он параметризует запросы и абстрагируется от разработчиков. Готовые шаблоны сами по себе защищают от атак с использованием межсайтовых сценариев.


\textbf{Защита паролей}

Используется функция защиты паролей PBKDF2.о функция получения ключа, разработанная
RSA Laboratories, используемая для получения стойких ключей на основе хеша. Она работает путем применения псевдослучайной хеш-функции к паролю
и повторение этого процесса большое количество раз.

\textbf{Защита от CSRF}

По умолчанию промежуточное ПО CSRF активировано в MIDDLEWARE настройках (CSRF защита теперь является частью ядра Django), а также можно использовать 
его методы для определенных представлений.

% \subsection{Actix (Rust)}
% \subsection{Revel (Gо)}

\newpage