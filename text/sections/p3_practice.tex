
\subsection{Критерии оценки безопасности фреймворков}
Может получится много ситуаций, когда приложение, его компоненты или язык на котором это все написано,
ведут себя непредсказуемо, это может быть связано с неправильной обработкой символов, отсутствием проверки каких-либо данных, 
передачи информации без надлежащей защиты ее или самого сеанса, поэтому попробую сформулировать некоторые критерии для фремворка web-приложения,
которые помогут оценить на сколько в плане безопасности продуман функционал.

% \textbf{Правильная обработка символов Юникода}
% Если такие символы, как пробелы и пунктуация, передаются в HTTP, может произойти ситуация, в которой на принимающей стороне
% символы могут быть истолкованы неверно. HTML-кодирование преобразует символы,
% которые не разрешены в HTML, в более менее подходящие аналоги, которые HTML поддерживает.


% \textbf{Влияние нулевых байтов}
% Возможна крайне неожиданная обработка нулевого байта. Строки, содержащие такой байт, могут обработаться только частично, до той позиции, в которой находится этот байт.

% \textbf{Поддержка параметризованных запросов к БД}
% Параметризированные запросы помогают избежать выдачи SQL-м данных из база, с помощью которого можно не допустить прямого ввода данных пользователем в запрос, а вместо этього
% использовать уже готовые шаблоны.

% \textbf{Защита от XSS}
% выполнения требуемой кодировки на основе местоположения вывода, чтобы предотвратить уязвимости межсайтового скриптинга? 
% (если нет, то шаблонизаторы, опасные теги, фильтрация тегов в подстановке html)

% \textbf{Безопасность реализации сеанса, доступ к ОРМ}
% Как происходит генерация токенов, возможно длительность сеанса. Сессия связи с бд, 
% защищенность.

% \textbf{Безопасность механизма аутентификации}

% Блок работы с бд:

% \textbf{Нормализация пути}

% \textbf{Безопасные варианты хранения}

% \textbf{Защита от инъекций строки для записи безопасных журналов}

% \textbf{Функция применения белого списка на входах}



\newpage
\subsection{ASP.NET (.NET)}


\newpage
\subsection{Laravel (Php)}


\newpage
\subsection{Django (Python)}
% \subsection{Actix (Rust)}
% \subsection{Revel (Gо)}


\newpage
\subsection{Сравнение защищенности}

\small
\centering
\begin{flushleft}
\begin{tabular}[t]{|p{9cm}|l|l|l|} \hline 
Фреймворки   Критерии& ASP.NET & Laravel & Django\\[3mm]\hline
Правильная обработка символов Юникода & + & + & + \\[2mm]\hline
Влияние нулевых байтов & + & +& + \\[2mm]\hline
Поддержка параметризованных запросов SQL & + & + & +\\[2mm]\hline
Наличие способа разделения данных и HTML & + & + & +\\[2mm]\hline
Безопасность реализации сеанса, безопасный доступ к ОРМ & + & + & +\\[2mm]\hline
Безопасность механизма аутентификации & + & + & +\\[2mm]\hline
\end{tabular}
\end{flushleft}

\centering
\begin{flushleft}
\begin{tabular}[t]{|p{9cm}|l|l|l|} \hline
Фреймворки  Критерии& ASP.NET & Laravel & Django\\[3mm]\hline
Нормализация пути & + & + & + \\[2mm]\hline
Безопасные варианты хранения & + & +& + \\[2mm]\hline
Защита от инъекций строки для записи безопасных журналовSQL & + & + & +\\[2mm]\hline
Функция применения белого списка на входах \newline шаблонизаторы, опасные теги & + & + & +\\[2mm]\hline
\end{tabular}
\end{flushleft}
